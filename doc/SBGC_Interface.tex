%----------------------------------------------------------------------------------------
%	PACKAGES AND OTHER DOCUMENT CONFIGURATIONS
%----------------------------------------------------------------------------------------


\documentclass[12pt]{article} % Default font size is 12pt, it can be changed here

\usepackage{geometry}
\geometry{left=22mm,right=22mm, top=30mm, bottom=25mm}
\usepackage{setspace} 
\usepackage{hyperref}
\usepackage[ngerman]{cleveref}

\usepackage{color,soul}

\usepackage{subfigure}


\usepackage{eurosym}

\usepackage{hhline,float}

\usepackage{fancyhdr}

\usepackage{german}

\usepackage{pdflscape}
\usepackage[utf8]{inputenc}
\usepackage[T1]{fontenc}


\usepackage{longtable}
 
\usepackage{geometry} % Required to change the page size to A4
\geometry{a4paper} % Set the page size to be A4 as opposed to the default US Letter

\usepackage{graphicx} % Required for including pictures

\usepackage{booktabs}

\usepackage{float} % Allows putting an [H] in \begin{figure} to specify the exact location of the figure
\usepackage{wrapfig} % Allows in-line images such as the example fish picture



\linespread{1.2} % Line spacing

%\setlength\parindent{0pt} % Uncomment to remove all indentation from paragraphs

\graphicspath{{Pictures/}} % Specifies the directory where pictures are stored



\begin{document}
	
	\begin{titlepage}
		
		\newcommand{\HRule}{\rule{\linewidth}{0.5mm}} % Defines a new command for the horizontal lines, change thickness here
		
		\centering % Center everything on the page
		
		\begin{figure}[h] 
			\centering
			\includegraphics[width=.4\textwidth]{Logo-FHNW}
		\end{figure}
		
		\textsc{\Large Zwischenbericht}\\[0.5cm] % Major heading such as course name
		\begin{doublespace}
			\HRule \\[1cm]
			{ \huge \bfseries Datenübertragung mit dem Pixhawk über die UART-Schnittstelle }\\[1cm] % Title of your document
			\HRule \\[1cm]
		\end{doublespace}

		{\large 17. Oktober 2016}\\[1cm] % Date, change the \today to a set date if you want to be precise
		
		{\large Saner Kevin}\\[1cm]
		{\large Institut für Automation}
		%\includegraphics{Logo}\\[1cm] % Include a department/university logo - this will require the graphicx package
		
		\vfill % Fill the rest of the page with whitespace
		
	\end{titlepage}
	
	\pagenumbering{arabic}
	\setcounter{page}{1}
	\pagestyle{fancy}
	\lfoot{18.10.2016} %left Foot
	\cfoot{\em Projekt FindMine} %Center Foot
	\rfoot{\thepage}
	
	\section{Einleitung}
	Die Antennen zum Suchen der Minen sollen mit einer Lagenregelung versehen werden, dazu wird auf ein Gimbal-System gesetzt. Die Konfiguration der Lagenregelung soll dabei vor dem Flug erfolgen. Das heisst, dass während dem Flug keine Änderungen vorgenommen werden müssen, jedoch soll die Winkel der Gimbal-Regelung überwacht werden, damit man sie allenfalls in das Postprocessing miteinbeziehen kann. Die im folgenden beschriebene Software, lässt also folglich nur Lese- und keine Schreibprozesse zu.  

	\begin{description}
		\item[Voraussetzungen:]~\par
		\begin{itemize}
			\item Ubuntu Version 14.04 oder ähnlich
			\item QGroundControl unter Linux
			\item Pixhawk inkl. sämtlicher benötigter Peripherie
			\item 2 x FTDI-Kabel 3.3V
			\item Toolchain des Single Board Computer zum Cross-Kompilieren
		\end{itemize}
	\end{description}
	

	
	

\newpage
\renewcommand\refname{Literaturverzeichnis}
\begin{thebibliography}{99} % Bibliography - this is intentionally simple in this template
	\raggedright
	
	\bibitem{Prosys}
	Prosys OPC UA Java SDK:
	\newblock {\em Preisliste}
	\newblock [online] Available at: https://downloads.prosysopc.com/opcua/Prosys\_OPC\_UA\_Java\_SDK\_Price\_List.pdf [Zugriff am 21.06.2016]

\end{thebibliography}

	
\end{document}